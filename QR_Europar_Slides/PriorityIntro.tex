% Slide: Priority Scheduling & Performance Trade-offs (Revised)
\begin{frame}{Priority Scheduling \& Performance Trade-offs}

  \begin{block}{Prioritizing the Critical Path}
    To further improve performance, we can execute more important tasks first.
    \begin{itemize}
      \item Inspired by the work of \textbf{Baskaran et al.}, we assign each task a priority.
      \item The priority is based on the task's \textbf{Bottom Level} (`bottomL`)—the longest path of work remaining from that task.
      \item This forces tasks on the \textbf{critical path} to be 
      scheduled sooner, potentially unlocking more parallelism.
    \end{itemize}
  \end{block}

  \begin{alertblock}{The Trade-off: Granularity vs. Overhead}
    The effectiveness of the priority queue depends on the task size $(\alpha, \beta)$:
    \begin{itemize}
      \item \textbf{Many small tasks:} High overhead from frequent queue rebalancing.
      \item \textbf{Fewer large tasks:} Low overhead, allowing the benefit of priority ordering to dominate.
    \end{itemize}
  \end{alertblock}

\end{frame}