% Slide: The Two-Queue Logic (Corrected and Redesigned)
\tikzstyle{startstop} = [rectangle, rounded corners, minimum width=3cm, minimum height=1cm, text centered, draw=black]
\tikzstyle{process} = [rectangle, minimum width=3cm, minimum height=1cm, text centered, draw=black]
\tikzstyle{decision} = [diamond, minimum width=3cm, minimum height=0.8cm, text centered, draw=black]
\tikzstyle{arrow} = [thick,->,>=stealth]

\begin{frame}{Our Solution: The Two-Queue Logic}
  
	\begin{columns}[c,onlytextwidth] % Vertically centered
			
		% ----- LEFT: The text explaining the diagram -----
		\pause 
		\column{0.45\textwidth}
		\begin{alertblock}{The Two-Queue Flow}
			The scheduler's logic is a simple loop:
			\begin{itemize}
		 \item A thread dequeues a task from the \textbf{Main Queue} and executes it.
		 \item Then it checks if the dependencies are satisfied for its child tasks.
		 \item \textbf{If YES:} The child task is enqueued in the main queue.
		 \item \textbf{If NO:} The child task is enqueued to the \textbf{Wait Queue} to be re-checked again by any thread.
			\end{itemize}
		\end{alertblock}
		
		% ----- RIGHT: The new, clean, and compact diagram -----
		\pause 
		\column{0.55\textwidth}
		% First Diagram
		\centering
		\begin{tikzpicture}[node distance=1.8cm, scale=0.7, transform shape, xshift=0cm]
	  
		 % Nodes
		 \node (root) [startstop, fill=red!30] {Root Node};
		 \node (mq) [process, below of=root, fill=blue!30] {Main Queue};
		 \node (executor) [process, below of=mq, fill=green!30] {Executor};
		 \node (dep) [decision, right of=executor, xshift=3cm, fill=yellow!30] {Dep Satisfied? };
		 \node (wq) [process, below of=dep, yshift=-2cm, fill=purple!30] {Wait Queue};
		 \node (terminate) [startstop, below of=executor, yshift=-2cm, fill=orange!30] {Terminate};
		 
		 % Arrows
		 \draw [arrow] (root) -- node[right] {Enqueue} (mq);
		 \draw [arrow] (mq) -- node[right] {Dequeue} (executor);
		 \draw [arrow] (executor.east) -- node[above] {If children} (dep.west);
		 \draw [arrow] (executor.south) -- node[right] {If no children} (terminate.north);
		 
		 % Rectangular arrow for "No" path
		 \draw [arrow] (dep.east) -- node[above] {No} ++(0.5,0) --node[midway,left] {Enqueue} ++(0,-3.8) -- (wq.east);
		 
		 % Arrow back from WQ
		 \draw [arrow] (wq.west) -- ++(-0.5,0) -- ++(-0.3, 0) -- node[midway,right] {Dequeue} ++(0.0, 2) -- ++(2.3,0) -- (dep.south);
		 
		 % "Yes" path looping back to MQ
		 \draw [arrow] (dep.north) |- node[right] {Yes} (mq.east) node[right, above]{\quad\quad\quad\quad\quad\quad\quad\quad Enqueue};  
		\end{tikzpicture}
	\end{columns}
\end{frame}