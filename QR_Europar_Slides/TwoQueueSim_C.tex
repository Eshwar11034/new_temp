% Simulation Slide 3, Step 1: The Initial State (Final Revised Version)
\begin{frame}{Simulation: Navigating the Dependency Graph}
  \begin{columns}[c,onlytextwidth]

    % ----- LEFT: Text for the initial state -----
    \column{0.38\textwidth}
    \only<1>{
      \begin{alertblock}{Mid-Execution Step}
      After executing $T_{2, 2}$ its children's dependencies are checked and enqueued to the main queue if its dependency is satisfied or else its enqueued into the wait queue.
        
      \end{alertblock}
    }
    \only<2>{
      \begin{alertblock}{Step 5: Dependency Aware Enqueue Process}
        Thread \textbf{T3} is processing its task from the previous level.
        So the children task $T_{2, 4}$ gets pushed to the wait queue as its parent task is incomplete.
      \end{alertblock}
    }
    \only<3>{
      \begin{alertblock}{Step 6: Continued Execution with Readily Available Tasks}
        The remaining threads can pick up other immediately executable tasks from the main queue.
      \end{alertblock}
    }
        \only<4>{
      \begin{alertblock}{Step 7: Next Pivot Push}
        \textbf{T1} finishes $T_{2, 3}$ and enqueues its child, the new pivot \textbf{$T_{3,3}$}. Also, $T_{1, 4}$ is complete in the mean time so its child $T_{2, 4}$ can be safely pushed to the main queue.
    \end{alertblock}
    }
        \only<5>{
      \begin{alertblock}{Step 8: Promotion Admist Work}
        Since, the parent of $T_{2, 4}$ was complete, $T_{2, 4}$ got promoted from the wait queue to the main queue as an immediately executable task.
      \end{alertblock}
    }

        \only<6>{
      \begin{alertblock}{Step 9: Further Execution with Immediately Executable Tasks}
        The threads can pick up more executable tasks from the main queue and continue executing the nodes of the DAG in a non-blocking manner.
      \end{alertblock}
    }


    % ----- RIGHT: The static diagram with your new layout and dashed arrows -----
    \column{0.61\textwidth}
    \centering
    \begin{tikzpicture}[font=\scriptsize, every node/.style={transform shape}, scale=0.95]
      % Styles
      \tikzset{pool/.style={draw, thick, ellipse, fill=accent!15, minimum width=2.8cm, minimum height=2cm}}
      \tikzset{thread_icon/.style={draw, circle, fill=accent!30, minimum size=0.5cm}}
      \tikzset{active_thread/.style={thread_icon, fill=accent!60, text=white, minimum size=0.6cm}}
      \tikzset{task_ready/.style={draw, rectangle, rounded corners=1pt, fill=updategreen!70, minimum width=1cm, minimum height=0.6cm}}
      \tikzset{task_wait/.style={draw, rectangle, rounded corners=1pt, fill=gray!50, minimum width=1cm, minimum height=0.6cm}}
      \tikzset{queue_box/.style={draw, thick, fill=gray!10, minimum width=5cm, minimum height=0.8cm}}
      
      % --- Layout Elements ---
      \node[pool] (pool) at (3.5, 0) {}; \node[anchor=north] at (pool.south) {Thread Pool};
      \node[queue_box] (mainq) at (-0.5, 2.1) {}; \node[anchor=south] at (mainq.north) {Main Queue};
      \node[font=\tiny] at ([yshift=-0.3cm]mainq.south west) {(Front)}; \node[font=\tiny] at ([yshift=-0.3cm]mainq.south east) {(Rear)};
      \node[queue_box] (waitq) at (-0.5, -2.1) {}; \node[anchor=north] at ([yshift=0.4cm]waitq.north) {Wait Queue};
      \node[font=\tiny] at ([yshift=-0.3cm]waitq.south west) {Front)}; \node[font=\tiny] at ([yshift=-0.3cm]waitq.south east) {(Rear)};
      
      % --- Initial State Elements ---
       \only<1>{
      % Active Threads (T3 and T4)
      \node[active_thread] (t3) at (0, 0.7) {\tiny T3}; \node[task_ready, right=1mm of t3] {$T_{1,4}$};
      \node[active_thread] (t4) at (0, -0.7) {\tiny T4}; \node[task_ready, right=1mm of t4] {$T_{1,5}$};
      
      \node[thread_icon] (t2_idle) at ($(pool.center)+(0,0.5)$) {\tiny T2};
      
      % Tasks in Main Queue
      \node[task_ready] (t22_q) at ([xshift=-1.8cm]mainq.center) {$T_{2,2}$};
          \node[active_thread] (active_t1) at (-2.5, -0.7) {\tiny T1};
        \draw[-{Stealth}, thick] (mainq.west) to[bend right=60] node[midway, left] {Dequeue} (active_t1);

      }


 % --- STATE 2 (After First Click - Corrected) ---


      % --- STATE 3 (After Second Click - Corrected) ---
      \only<2>{
        % T3 is still active, executing the parent task
        \node[active_thread] (t3) at (0, 0.7) {\tiny T3}; \node[task_ready, right=1mm of t3] {$T_{1,4}$};
        \node[active_thread] (active_t2) at (-2.5, -0.7) {\tiny T1};
        % The fail arrow appears, and T2,4 moves to the Wait Queue
    
        \node[task_wait] (t24_q)at ([xshift=1.8cm]waitq.center) {$T_{2,4}$};
        \draw[-{Stealth}, thick, depred, dashed] (active_t2) to[out=180, in=220] 
        node[midway, below left] {Fail} (t24_q.south);
        \node[task_ready] (t23_q) at ([xshift=0.6cm]mainq.west) {$T_{2,3}$};
        \node[task_ready] (t25_q) at ([xshift=1.6cm]mainq.west) {$T_{2,5}$};
          \node[thread_icon] (t4_idle) at ($(pool.center)+(0,-0.5)$) {\tiny T4};
        \node[thread_icon] (t2_idle) at ($(pool.center)+(0,0.5)$) {\tiny T2};   
         \draw[-{Stealth}, thick, dashed] (active_t1) to[bend right=60] 
        node[midway, below right] {Pass} (t25_q.east);

      }
      % --- STATE 4 (After Third Click) ---
      \only<3>{
        \node[active_thread] (t3) at (0, 0.7) {\tiny T3}; \node[task_ready, right=1mm of t3] {$T_{1,4}$};
        \node[thread_icon] at ($(pool.center)+(0.8,0.2)$) {\tiny T4};
       
        \node[active_thread] (active_t1) at (-2.5, -0.7) {\tiny T1};
        \draw[-{Stealth}, thick] (mainq.west) to[bend right=60] node[midway, left] {Dequeue} (active_t1);

        \node[active_thread] (active_t2) at (-1.5, -0.7) {\tiny T2};
        
        \node[task_ready] (t23_q) at ([xshift=0.6cm]mainq.west) {$T_{2,3}$};

        \node[task_ready] (t25_q) at ([xshift=1.6cm]mainq.west) {$T_{2,5}$};
    
        % T2,4 is still in the Wait Queue
        \node[task_wait] at ([xshift=1.8cm]waitq.center) {$T_{2,4}$};
        
      \draw[-{Stealth}, thick] (t25_q.east) to[bend left=60]  (active_t2);
      }
      % --- STATE 5 (After Fourth Click) ---
      \only<4>{
       

        \node[thread_icon] at ($(pool.center)+(0.8,0.2)$) {\tiny T3};
        \node[thread_icon] at ($(pool.center)+(-0.5,-0.5)$) {\tiny T4};
        \node[active_thread] (active_t2) at (-2.5, -0.7) {\tiny T2};
        \node[task_ready, right=1mm of active_t2] {$T_{2,5}$};

        \node[task_ready] at ([xshift=-0.8cm]mainq.east) {$T_{3,3}$};

        \node[active_thread] (t3) at (0, 0.7) {\tiny T1}; 
        \draw[-{Stealth}, thick] (0.3, 0.7) to[bend right=30] node[midway, right] {enqueue}([xshift=1.4cm]mainq.south);
        
        \node[task_ready] at ([xshift=0.8cm]waitq.west) {$T_{2,4}$};
      }

      \only<5>{
        % T1 and T3 are idle

        \node[thread_icon] at ($(pool.center)+(0.8,0.2)$) {\tiny T4};
        \node[thread_icon] at ($(pool.center)+(0.1,0.2)$) {\tiny T2};

        \node[active_thread] (active_t2) at (-2.5, -0.7) {\tiny T1};

        \draw[-{Stealth}, thick, updategreen, dashed] (waitq.west) to[bend left =45] (active_t2);
        \draw[-{Stealth}, thick, updategreen, dashed] (active_t2) to[bend right=30] node[midway, left] {Promote} ([xshift = 1.6cm]mainq.south);
        \node[task_ready] at ([xshift=1.8cm]mainq.center) {$T_{2,4}$};

      \node[active_thread] (t4) at (0, -0.7) {\tiny T3}; \node[task_ready, right=1mm of t4] {$T_{3,3}$};
      }

      % --- STATE 7 (After Sixth Click) ---
    \only<6->{
        % Idle threads remaining in the pool
        \node[thread_icon] at ($(pool.center)+(-0.8,0.2)$) {\tiny T3};
        
        % T2 and T4 are now active, taking the NEW work
        \node[active_thread] (active_t2) at (-2.5, -0.7) {\tiny T1};
        \node[active_thread] (active_t1) at (0, -0.7) {\tiny T4};
        \node[active_thread] (active_t3) at (-3.5, 0.7) {\tiny T2};
        
        % Arrows showing T2 and T4 dequeuing the new tasks
        \draw[-{Stealth}, thick] ([xshift=0.4cm]mainq.center) to[bend right = 15] (active_t2);
        \draw[-{Stealth}, thick] ([xshift=1.8cm]mainq.center) to[bend right] (active_t1);
        \draw[-{Stealth}, thick] ([xshift=0.4cm]mainq.west) to[bend right = 45] (active_t3);
        % T3 has implicitly finished T3,3 and enqueued its children
        % The children are now in the main queue
        \node[task_ready] at ([xshift=0.4cm]mainq.center) {$T_{3,4}$};
        \node[task_ready] at ([xshift=1.8cm]mainq.center) {$T_{3,5}$};
        
        % T2,4 is also still in the queue, ready to be picked up
        \node[task_ready] at ([xshift=-1.8cm]mainq.center) {$T_{2,4}$};
        
        }
      
\end{tikzpicture}
\end{columns}
\end{frame}