% Final Slide: Defining alpha and beta visually with annotations
\begin{frame}{Tuning Task Granularity with $\alpha$ and $\beta$}
  \begin{columns}[T,onlytextwidth]

    % ----- LEFT: The approved, paper-based text -----
    \column{0.6\textwidth}
    \begin{block}{The Problem: High Scheduling Overhead}
      The original task DAG contains many small, fine-grained tasks. Managing these individually creates significant scheduling overhead.
    \end{block}

    \begin{alertblock}{Our Solution: Coalescing Tasks}
      We introduce two control parameters to group the fine-grained tasks into larger, more efficient chunks of work:
      \begin{itemize}
        \item \textbf{$\alpha$ (Alpha):} Controls the number of pivot computations that are performed as a single batch.
        \item \textbf{$\beta$ (Beta):} Determines how many trailing rows are grouped together for a thread to update simultaneously.
      \end{itemize}
    \end{alertblock}

    % ----- RIGHT: The visual explanation using the Task Graph -----
    \column{0.35\textwidth}
    \centering
    \definecolor{lightblue}{RGB}{173, 216, 230}
    % You MUST add this to your main.tex preamble: \usetikzlirary{fit}
    \begin{tikzpicture}[node distance=1cm and 0.5cm, scale=0.5, transform shape]
      % --- Base DAG Nodes ---
      \node[draw, circle, fill=lightblue] (T11) at (0,0) {$T_{1,1}$};
      \node[draw, circle, fill=lightblue] (T12) at (1.4,-1.2) {$T_{1,2}$};
      \node[draw, circle] (T13) at (3.2,-1.2) {$T_{1,3}$};
      \node[draw, circle] (T14) at (4.8,-1.2) {$T_{1,4}$};
      \node[draw, circle] (T15) at (6.4,-1.2) {$T_{1,5}$};
      \node[draw, circle, fill=lightblue] (T22) at (1.4,-2.5) {$T_{2,2}$};
      \node[draw, circle, fill=lightblue] (T23) at (3.2,-3.5) {$T_{2,3}$};
      \node[draw, circle] (T24) at (4.8,-3.5) {$T_{2,4}$};
      \node[draw, circle] (T25) at (6.4,-3.5) {$T_{2,5}$};
      \node[draw, circle, fill=lightblue] (T33) at (3.2,-4.8) {$T_{3,3}$};
      \node[draw, circle, fill=lightblue] (T34) at (4.8,-5.8) {$T_{3,4}$};
      \node[draw, circle] (T35) at (6.4,-5.8) {$T_{3,5}$};
      \node[draw, circle, fill=lightblue] (T44) at (4.8,-7.1) {$T_{4,4}$};
      \node[draw, circle, fill=lightblue] (T45) at (6.4,-8.1) {$T_{4,5}$};
      \node[draw, circle, fill=lightblue] (T55) at (6.4,-9.7) {$T_{5,5}$};
      
      % --- Base DAG Edges ---
      \draw[->] (T11) -- (T12); \draw[->] (T12) -- (T22); \draw[->] (T13) -- (T23);
      \draw[->] (T14) -- (T24); \draw[->] (T15) -- (T25); \draw[->, bend left] (T11) to (T13);
      \draw[->, bend left] (T11) to (T14); \draw[->, bend left] (T11) to (T15);
      \draw[->] (T22) -- (T23); \draw[->] (T24) -- (T34); \draw[->] (T25) -- (T35);
      \draw[->, bend left] (T22) to (T24); \draw[->, bend left] (T22) to (T25);
      \draw[->] (T23) -- (T33); \draw[->, bend left] (T33) to (T34); \draw[->, bend left] (T33) to (T35);
      \draw[->] (T34) -- (T44); \draw[->] (T35) -- (T45); \draw[->, bend left] (T44) to (T45);
      \draw[->] (T45) -- (T55);

      % --- Alpha Visualization (Red Ellipse) ---
      \node[draw, red, thick, ellipse,, xscale=0.8, inner sep=0pt, fit=(T11) (T12) (T22), label={[red]left:$\alpha$-group}] {};
      
      % --- Beta Visualization (Blue Ellipse) ---
      \node[draw, blue, thick, ellipse, xscale=0.8, inner sep=2pt, fit=(T13) (T14) (T15), label={[blue,yshift=0.2cm]above:$\beta$-group}] {};
    \end{tikzpicture}
    
    \vspace{2mm} % Add some space below the diagram
    
    % --- Your requested annotations, placed below the figure ---
    \begin{flushleft}
    \tiny % Use tiny font to ensure it fits
    \textcolor{red}{\textbf{$\alpha$-group:}} Groups \textbf{dependent} tasks along the critical path. \\
    \textcolor{blue}{\textbf{$\beta$-group:}} Groups tasks that can be \textbf{parallelized}.
    \end{flushleft}
    
  \end{columns}
\end{frame}