% Slide: Barrier-Based Task Scheduling
\begin{frame}{A Simple Approach: Barriers on a Task Graph}
  \begin{columns}[c,onlytextwidth] % Vertically center the columns

    % ----- LEFT: Explanation of the barrier logic for tasks -----
    \column{0.50\textwidth}
    \begin{block}{The Barrier-Based Approach}
      A straightforward way to parallelize the QR algorithm is to first model it as a graph of tasks ($T_{i,j}$) and then use barriers to enforce dependencies:
      \begin{itemize}
        \item \textbf{Barrier 1:} After the pivot task \textcolor{pivotblue}{$T_{i,i}$} completes, all threads must wait. This ensures the transformation info is available.
        \item \textbf{Barrier 2:} After all parallel update tasks \textcolor{updategreen}{$T_{i,j}$} are done, all threads must wait again before starting the next iteration ($i+1$).
      \end{itemize}
    \end{block}
    


    % ----- RIGHT: The TikZ figure illustrating the problem -----
    \column{0.45\textwidth}
          \begin{alertblock}{The Critical Flaw: Wasted Time}
      Barriers impose a rigid, synchronous execution. Threads that finish their tasks early are forced to \textbf{idle}, waiting for the single \textbf{slowest thread} to catch up. This "tail waiting" is pure computational waste.
    \end{alertblock}
    % --- Rescaled and re-dimensioned the TikZ figure to prevent overflow ---
    \begin{tikzpicture}[font=\tiny, every node/.style={transform shape}, scale=0.60]
      % Styles for different states
      \tikzset{pivot/.style={fill=pivotblue!70, draw=pivotblue!50!black, rounded corners=1pt}}
      \tikzset{work/.style={fill=updategreen!70, draw=updategreen!50!black, rounded corners=1pt}}
      \tikzset{wait/.style={fill=gray!40, draw=gray!60, rounded corners=1pt}}
      \tikzset{bar/.style={draw=depred, thick, dashed}}

      % Time Axis
      \draw[->] (0,0) -- (9.5,0) node[anchor=north west, font=\scriptsize] {Time};
      
      % Thread Labels
      \node[anchor=east, font=\scriptsize] at (-0.2, 2.5) {Thread 1};
      \node[anchor=east, font=\scriptsize] at (-0.2, 1.9) {Thread 2};
      \node[anchor=east, font=\scriptsize] at (-0.2, 1.3) {Thread 3};
      \node[anchor=east, font=\scriptsize] at (-0.2, 0.7) {Thread 4};

      % --- Iteration i with new, narrower coordinates ---
      \draw[pivot] (0.2, 2.3) rectangle node{$T_{i,i}$} (1.2, 2.7);
      \draw[wait]  (0.2, 1.7) rectangle (1.2, 2.1);
      \draw[wait]  (0.2, 1.1) rectangle (1.2, 1.5);
      \draw[wait]  (0.2, 0.5) rectangle (1.2, 0.9);
      \node[above=2pt, depred, font=\scriptsize] at (1.3, 2.8) {Barrier 1};
      \draw[bar] (1.3, 0.3) -- (1.3, 2.8);

      \draw[work] (1.4, 2.3) rectangle node{$T_{i,j}$} (4.5, 2.7);
      \draw[work] (1.4, 1.7) rectangle node{$T_{i,k}$} (6.5, 2.1);
      \draw[work] (1.4, 1.1) rectangle node{$T_{i,l}$} (8.0, 1.5); % SLOWEST
      \draw[work] (1.4, 0.5) rectangle node{$T_{i,m}$} (3.5, 0.9);

      \draw[wait] (4.5, 2.3) rectangle node{\textcolor{black}{IDLE}} (8.0, 2.7);
      \draw[wait] (6.5, 1.7) rectangle node{\textcolor{black}{IDLE}} (8.0, 2.1);
      \draw[wait] (3.5, 0.5) rectangle node{\textcolor{black}{IDLE}} (8.0, 0.9);
      
      \node[above=2pt, depred, font=\scriptsize] at (8.1, 2.8) {Barrier 2};
      \draw[bar] (8.1, 0.3) -- (8.1, 2.8);

      \draw[pivot] (8.2, 2.3) rectangle node{$T_{i+1,i+1}$} (9.2, 2.7);
    \end{tikzpicture}
    
    % --- Added the requested figure caption ---
    \vspace{2mm}
    {\scriptsize \emph{Barrier scheduling forces fast threads to wait, causing idle time.}}
  \end{columns}
\end{frame}