% Slide: What The Kernels Actually Do (Final)
\begin{frame}{What Do The Kernels Actually Do?}
  \begin{columns}[T,onlytextwidth]

    % ----- LEFT: Stage 1 - Finding the Transformation -----
    \column{0.45\textwidth}
    \begin{block}{1. The Pivot Kernel's Job}
\begin{itemize}
    \item \textbf{Define Transformation:} Calculate the pivot row's vector norm which is  defined a ``reflection`` transformation, which is used to introduce zeros into the column.
    \item \textbf{Modify \& Output:} Modify the pivot row's diagonal element and output the parameters that describe the transformation.
\end{itemize}
    \end{block}
    
    \centering
    \begin{tikzpicture}[font=\large, every node/.style={transform shape}, scale=0.9]
        % Before
        \node (before) at (-3.5, 0) {$
            \begin{pmatrix} d & c & g \end{pmatrix}
        $};
        % After
        \node (after) at (1.5, 0) {$
            \begin{pmatrix} d' & \mathbf{0} & \mathbf{0} \end{pmatrix}
        $};
        % The operation arrow
        \draw[-{Stealth}, pivotblue, very thick] (before) -- node[above, font=\tiny] {\texttt{update\_pivot\_row}} (after);

    \end{tikzpicture}
    
    % ----- RIGHT: Stage 2 - Applying the Transformation -----
    \column{0.5\textwidth}
    \begin{block}{2. The Update Kernel's Job}
    It takes the transformation parameters that were previously generated by the pivot kernel and applies this transformation to the trailing rows of the matrix.
    \end{block}

    \centering
    \begin{tikzpicture}[font=\large, every node/.style={transform shape}, scale=0.95]
        % Before
        \node (before1) at (-2.5, 4.0) {$
            \begin{pmatrix} d' & \mathbf{0} & \mathbf{0} \end{pmatrix}
        $};
        \node (before2) at (-2.5, 3.0) {$
            \begin{pmatrix} a & e & h \end{pmatrix}
        $};
        \node (before3) at (-2.5, 2.0) {$
            \begin{pmatrix} b & c & f \end{pmatrix}
        $};
        % After
        \node (after1) at (3.5, 4.0) {$
            \begin{pmatrix} d' & \mathbf{0} & \mathbf{0} \end{pmatrix}
        $};
        \node (after2) at (3.5, 3.0) {$
            \begin{pmatrix} a'' & e'' & h'' \end{pmatrix}
        $};
        \node (after3) at (3.5, 2.0) {$
            \begin{pmatrix} b'' & c'' & f'' \end{pmatrix}
        $};
        % The operation arrow
        \draw[-{Stealth}, updategreen, very thick] (before2) -- node[above, font=\tiny] {\texttt{update\_trailing\_non\_pivot\_row}} (after2);
    \end{tikzpicture}

  \end{columns}
\end{frame}