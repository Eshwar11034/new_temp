\begin{frame}{Simulation: The Two-Queue Scheduler in Action}
  \begin{columns}[c,onlytextwidth] % Vertically centered

    % ----- LEFT: The text explaining State B -----
    \column{0.4\textwidth}
    \begin{alertblock}{State B: Execution \& Spawning}
      An idle thread, \textbf{T1}, dequeues the pivot task \textbf{$T_{1,1}$} from the \textbf{front} of the Main Queue.
      \vspace{2mm}
      
      After executing it, T1 enqueues the newly spawned, ready-to-run child tasks (\textbf{$T_{1,2}, T_{1,3}, \dots$}) to the \textbf{rear} of the Main Queue.
    \end{alertblock}

    % ----- RIGHT: The diagram showing the action of State B -----
    \column{0.6\textwidth}
    \centering
    \begin{tikzpicture}[font=\scriptsize, every node/.style={transform shape}, scale=0.95]
      % Styles
      \tikzset{pool/.style={draw, thick, ellipse, fill=accent!15, minimum width=2.8cm, minimum height=2cm, align=center}}
      \tikzset{thread_icon/.style={draw, circle, fill=accent!30, minimum size=0.5cm}}
      \tikzset{active_thread/.style={thread_icon, fill=accent!60, text=white, minimum size=0.6cm}}
      \tikzset{task_ready/.style={draw, rectangle, rounded corners=1pt, fill=updategreen!70, minimum width=1cm, minimum height=0.6cm}}
      \tikzset{queue_box/.style={draw, thick, fill=gray!10}}
      
      % --- Repositioned Elements for Clarity ---
      % Main Queue at the top
      \node[anchor=south] at (-0.5, 2.5) {Main Queue};
      \draw[queue_box] (-3, 1.8) rectangle (2, 2.4);
      \node[font=\tiny] at (-3, 1.5) {(Front)};
      \node[font=\tiny] at (2, 1.5) {(Rear)};

      % Wait Queue at the bottom
      \node[anchor=south] at (-0.5, -1.5) {Wait Queue};
      \draw[queue_box] (-3, -2.2) rectangle (2, -1.6);
      \node[font=\tiny] at (-3, -2.5) {(Front)};
      \node[font=\tiny] at (2, -2.5) {(Rear)};

      % Thread Pool on the right (T1 is missing)
      \node[pool] (pool) at (3.5, 0) {};
      \node[thread_icon] at ($(pool.center)+(0.4,0.3)$) {\tiny T2};
      \node[thread_icon] at ($(pool.center)+(-0.5, 0.4)$) {\tiny T3};
      \node[thread_icon] at ($(pool.center)+(-0.2, -0.5)$) {\tiny T4};
      \node[anchor=north] at (pool.south) {Thread Pool};

      % --- Active T1 and its task are in the center-left "workspace" ---
      \node[active_thread] (active_t1) at (-2.5, -0.2) {\tiny T1};
      \node[task_ready] (task_t11) at (-1, 0.5) {$T_{1,1}$};

      % --- The Actions (Clear, Curved Arrows) ---
      % Dequeue Arrow: from Front of Queue to the active T1
      \draw[-{Stealth}, thick] (-3, 2.1) to[bend right=45] node[midway, left] {Dequeue} (active_t1);
      
      % Enqueue Arrow: from the processed task to the Rear of Queue
      \draw[-{Stealth}, thick, dashed] (task_t11) to[bend left=20] node[midway, right] {Enqueue} (2, 2.1);
      
      % --- The Result: New tasks are now in the Main Queue ---
      \node[task_ready] at (1.5, 2.1) {$T_{1,4}$};
      \node[task_ready] at (0.4, 2.1) {$T_{1,3}$};
      \node[task_ready] at (-0.7, 2.1) {$T_{1,2}$};
      
    \end{tikzpicture}
  \end{columns}
\end{frame}