% Slide 11 — Two-Queue Scheduler: Flowchart (from source) + Pseudocode (smaller)
\begin{frame}{Two-Queue Scheduler — Flow \& Pseudocode}
\footnotesize
\begin{columns}[T,onlytextwidth]

  % -------- LEFT: pseudocode (compact) --------
  \column{0.50\textwidth}
\begin{block}{Pseudocode (ready-first, no global barriers)}
  % --- FONT SIZE DECREASED TO MAKE BLOCK MORE COMPACT ---
  \footnotesize
  \begin{algorithmic}[1]
    \State \textbf{Queues:} main (ready-first), wait
    \While{true}
      \State $t \gets$ pop(main)
      \If{$t = \varnothing$}
        \State promote\_ready(wait $\rightarrow$ main)
        \If{empty(main) \textbf{and} empty(wait)} \textbf{break} \EndIf
        \State \textbf{continue}
      \EndIf
      \If{\textbf{not} parents\_done($t$)} \Comment{e.g., missing $T_{i-1,j}$}
        \State push(wait, $t$) \Comment{dequeued but not ready}
        \State \textbf{continue}
      \EndIf
      \State execute($t$)
      \ForAll{$c \in$ children($t$)}
        \State push(main, $c$) \Comment{spawn to main (ready-first)}
      \EndFor
    \EndWhile
  \end{algorithmic}
\end{block}

  \vspace{1mm}
  \begin{block}{Readiness for $T_{i,j}$}
    parents\_done$(T_{i,j}) \iff T_{i,i}$ and $T_{i-1,j}$ complete \;(ignore $T_{0,j}$).
  \end{block}

  % -------- RIGHT: flowchart (ripped from your source, scaled down) --------
  \column{0.50\textwidth}
  % Styles as in methodology.tex
  \tikzstyle{startstop} = [rectangle, rounded corners, minimum width=2.2cm, minimum height=0.8cm, text centered, draw=black]
  \tikzstyle{process}   = [rectangle, minimum width=2.2cm, minimum height=0.8cm, text centered, draw=black]
  \tikzstyle{decision}  = [diamond,   minimum width=2.2cm, minimum height=0.7cm, text centered, draw=black, aspect=2]
  \tikzstyle{arrow}     = [thick,->,>=stealth]

  \centering
  \begin{tikzpicture}[node distance=1.3cm, scale=0.58, transform shape]
    % Nodes (as in your methodology.tex)
    \node (root)      [startstop, fill=red!30]                 {Root Node};
    \node (mq)        [process,   below of=root, fill=blue!30] {Main Queue};
    \node (executor)  [process,   below of=mq,   fill=green!30]{Executor};
    \node (dep)       [decision,  right of=executor, xshift=2.6cm, fill=yellow!30] {Dep Satisfied?};
    \node (wq)        [process,   below of=dep, yshift=-1.6cm,  fill=purple!30] {Wait Queue};
    \node (terminate) [startstop, below of=executor, yshift=-1.6cm, fill=orange!30] {Terminate};

    % Arrows (ported; two long paths reconstructed where upload was truncated)
    \draw [arrow] (root) -- node[right] {Enqueue} (mq);
    \draw [arrow] (mq) -- node[right] {Dequeue} (executor);
    \draw [arrow] (executor.east) -- node[above] {If children} (dep.west);
    \draw [arrow] (executor.south) -- node[right] {If no children} (terminate.north);

    % "No" path: dep -> wait queue (exact from source form)
    \draw [arrow] (dep.east) -- node[above] {No} ++(0.5,0)
                  -- node[midway,left] {Enqueue} ++(0,-3.0) -- (wq.east);

    % Return from wait queue to dep (reconstructed per source intent)
    \draw [arrow] (wq.west) -- ++(-0.5,0) -- ++(-0.3,0)
                  -- node[midway,right] {Dequeue} ++(0,1.6) -- ++(2.1,0) -- (dep.south);

    % "Yes" path: dep -> main queue (ported; label retained)
    \draw [arrow] (dep.north) |- node[right] {Yes} (mq.east);

  \end{tikzpicture}

\end{columns}
\end{frame}
