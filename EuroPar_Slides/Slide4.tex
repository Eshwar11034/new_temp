% Slide 4 — QR as a Task DAG (Problem Statement, with paper figure replicated in TikZ)
\begin{frame}{QR as a Task DAG — Problem Statement}
\small
\begin{columns}[T,onlytextwidth]

  % ---------- LEFT: context & problem ----------
  \column{0.52\textwidth}
  \begin{block}{Tasks and what they represent}
    \begin{tightitem}
      \item In-place Householder QR is modeled as tasks $T_{i,j}$ over the upper triangle ($1\!\le\!i\!\le\!j\!\le\!n$).

      \item $\mathrm{parents}(T_{i,j})=\{\,T_{i,i}\text{ (pivot)},\;T_{i-1,j}\text{ (prev.\ row)}\,\}$ .
      \item The diagonal $\{T_{1,1},T_{2,2},\ldots\}$ is the \textbf{critical path}; finishing it unlocks width.
      \item A task is \emph{ready} to be executed once all its parents complete execution.
    \end{tightitem}
  \end{block}

  \vspace{0.5mm}
  \begin{block}{Scheduling goal}
    \begin{tightitem}
      \item Execute this DAG on $p$ threads \textbf{without global barriers}, keeping workers on \emph{ready} tasks and releasing the critical path early—while preserving the in-place interface SLSQP needs.
    \end{tightitem}
  \end{block}

    % ---- Right: TaskGraph figure (ported from paper’s TikZ) ----
    \column{0.50\textwidth}
    \centering
    % color from the paper
    \definecolor{lightblue}{RGB}{173,216,230}
    \begin{tikzpicture}[node distance=1cm and 0.5cm, scale=0.55, transform shape]
      % Nodes - Row 1
      \node[draw, circle, fill=lightblue] (T11) at (0,0) {$T_{1,1}$};
      \node[draw, circle, fill=lightblue] (T12) at (1.4,-1.2) {$T_{1,2}$};
      \node[draw, circle] (T13) at (2.6,-1.2) {$T_{1,3}$};
      \node[draw, circle] (T14) at (3.8,-1.2) {$T_{1,4}$};
      \node[draw, circle] (T15) at (4.9,-1.2) {$T_{1,5}$};

      % Nodes - Row 2
      \node[draw, circle, fill=lightblue] (T22) at (1.4,-2.5) {$T_{2,2}$};
      \node[draw, circle, fill=lightblue] (T23) at (2.6,-3.5) {$T_{2,3}$};
      \node[draw, circle] (T24) at (3.8,-3.5) {$T_{2,4}$};
      \node[draw, circle] (T25) at (4.9,-3.5) {$T_{2,5}$};

      % Nodes - Row 3
      \node[draw, circle, fill=lightblue] (T33) at (2.6,-4.8) {$T_{3,3}$};
      \node[draw, circle, fill=lightblue] (T34) at (3.8,-5.8) {$T_{3,4}$};
      \node[draw, circle] (T35) at (4.9,-5.8) {$T_{3,5}$};

      % Nodes - Row 4
      \node[draw, circle, fill=lightblue] (T44) at (3.8,-7.1) {$T_{4,4}$};
      \node[draw, circle, fill=lightblue] (T45) at (4.9,-8.1) {$T_{4,5}$};

      % Nodes - Row 5
      \node[draw, circle, fill=lightblue] (T55) at (4.9,-9.7) {$T_{5,5}$};

      % Edges (ported from paper)
      \draw[->] (T11) -- (T12);
      \draw[->] (T12) -- (T22);
      \draw[->] (T13) -- (T23);
      \draw[->] (T14) -- (T24);
      \draw[->] (T15) -- (T25);
      \draw[->, bend left] (T11) to (T13);
      \draw[->, bend left] (T11) to (T14);
      \draw[->, bend left] (T11) to (T15);

      \draw[->] (T22) -- (T23);
      \draw[->] (T24) -- (T34);
      \draw[->] (T25) -- (T35);
      \draw[->, bend left] (T22) to (T24);
      \draw[->, bend left] (T22) to (T25);

      \draw[->] (T23) -- (T33);
      \draw[->, bend left] (T33) to (T34);
      \draw[->, bend left] (T33) to (T35);

      \draw[->] (T34) -- (T44);
      \draw[->] (T35) -- (T45);
      \draw[->, bend left] (T44) to (T45);

      \draw[->] (T45) -- (T55);
    \end{tikzpicture}

    \vspace{1mm}
    {\scriptsize \emph{“TaskGraph for Triangular System”}}

  \end{columns}
\end{frame}
