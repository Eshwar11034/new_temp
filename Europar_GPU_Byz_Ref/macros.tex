% comment a region
% Put all your custom macros here
\usepackage{xspace}
\newcommand{\cmnt}[1]{}
\newcommand{\ignore}[1]{}
\newcommand{\remove}[1]{}

% a word should not be broken across lines
\newcommand{\nosplit}{\linebreak}

% no hyphenation
\def\nohyphens{\hyphenpenalty=10000\exhyphenpenalty=10000}

% ~ character
\newcommand{\tilda}{\symbol{126}}

% useful mathematical symbols

\newcommand{\ang}[1]{\langle #1 \rangle}
\newcommand{\Ang}[1]{\Big\langle #1 \Big\rangle}
\newcommand{\ceil}[1]{\lceil #1 \rceil}
\newcommand{\floor}[1]{\lfloor #1 \rfloor}
% \newtheorem{definition}{Definition}

% quantifiers
\newcommand{\myforall}[3]{\ang{\forall \: #1 : #2 : #3 }}
\newcommand{\myexists}[3]{\ang{\exists \: #1 : #2 : #3 }}
\newcommand{\mynexists}[3]{\ang{\nexists \: #1 : #2 : #3 }}
\newcommand{\Myforall}[3]{\Ang{\forall \: #1 : #2 : #3 }}
\newcommand{\Myexists}[3]{\Ang{\exists \: #1 : #2 : #3 }}
\newcommand{\Mynexists}[3]{\Ang{\nexists \: #1 : #2 : #3 }}


% for formally defining a notion
% \newcommand{\defined}{\;\;\stackrel{\triangle}{=}\;\;}
\newcommand{\defined}{\;\;\triangleq\;\;}


% mathematical implications

%\renewcommand{\implies}{\Rightarrow}
\newcommand{\follows}{\Leftarrow}

% miscellaenous
\newcommand{\myspace}{\hspace*{0.0in}}
\newcommand{\halflinespacing}{\vspace*{0.5em}}
\newcommand{\greaterlinespacing}{\vspace*{0.75em}}

%/commands for personal comments
\newcommand {\spnote}[1] {\todo[inline,size=\footnotesize,color=yellow!20]{Sathya: #1}}
\newcommand {\manasnote}[1] {\todo[inline,size=\footnotesize,color=yellow!20]{Manas: #1}}
\newcommand {\cgnote}[1] {\todo[inline,size=\footnotesize,color=yellow!20]{Chryssis: #1}}

\newcommand {\spcolor}[1] {\textcolor{green}{#1}}
\newcommand {\spdiff}[1] {\textcolor{orange}{#1}}
\newcommand {\mpcolor}[1] {\textcolor{blue}{#1}}

%%%%%%%%%%%%%%%%%%%%%%%%%%%%%%%%%%%%%%%%%%%%%%%%%%%%%%%%%%%%%%%%%%%%%%%%%%%%%%%%%%%%%%%%%%%%%%

%% various theorem environments
%% Commented out because they are already defined in llncs file %%
%% Decommented out because they are not defined in sigplan class %%

% \newtheorem{theorem}{Theorem}
%\renewcommand{\thetheorem}{\arabic{theorem}}
% \newtheorem{lemma}{Lemma}

%\newtheorem{lemma}[theorem]{Lemma}
% \newtheorem{corollary}{Corollary}
%\newtheorem{proposition}[theorem]{Proposition}
%\newtheorem{property}[theorem]{Property}
%\newtheorem{definition}[theorem]{Definition}

%\renewtheorem{property}[theorem]{Property}

% \newtheorem{definition}{Definition}
%\renewcommand{\thedefinition}{\arabic{definition}}

%\newtheorem{property}{Property}
%\renewcommand{\theproperty}{\arabic{property}}

% \newtheorem{observation}[lemma]{Observation}
%\renewcommand{\theobservation}{\arabic{observation}}

%\newtheorem{example}{Example}
%\renewcommand{\theexample}{\arabic{example}}

%\newtheorem{assume}{Assumption}
%\renewcommand{\theassume}{\arabic{assume}}

%\newtheorem{remark}{Remark}
%\renewcommand{\theremark}{\arabic{remark}}

%\newtheorem{axiom}[theorem]{Proposition}
%\renewcommand{\theaxiom}{\arabic{axiom}}

%\newtheorem{invariant}[theorem]{Invariant}

%\renewcommand{\theequation}{\arabic{equation}}

%-----------------Theorem Notations Defined by Petr ---------------------------------------------------------------------------------------------------------
%% Defined by Petr
\newtheorem{requirement}{Requirement}
%------------------------------------------------------------------------------------------------------------------------------

% \newtheorem{definition}{Definition}

% various references
\newcommand{\chapref}[1]{Chapter~\ref{chap:#1}}
\newcommand{\secref}[1]{Section~\ref{sec:#1}}
\newcommand{\figref}[1]{Fig.~\ref{fig:#1}}
\newcommand{\tabref}[1]{Table~\ref{tab:#1}}
\newcommand{\stref}[1]{step~\ref{step:#1}}
\newcommand{\thmref}[1]{Theorem~\ref{thm:#1}}
\newcommand{\lemref}[1]{Lemma~\ref{lem:#1}}
%\newcommand{\corref}[1]{Corollary~\ref{cor:#1}}
\newcommand{\axmref}[1]{Proposition~\ref{axm:#1}}
\newcommand{\defref}[1]{Definition~\ref{def:#1}}
\newcommand{\eqnref}[1]{Eqn(\ref{eq:#1})}
\newcommand{\eqvref}[1]{Equivalence~(\ref{eqv:#1})}
\newcommand{\ineqref}[1]{Inequality~(\ref{ineq:#1})}
%\newcommand{\invref}[1]{(\ref{inv:#1})}
\newcommand{\exref}[1]{Example~\ref{ex:#1}}
\newcommand{\propref}[1]{Property~\ref{prop:#1}}
\newcommand{\obsref}[1]{Observation~\ref{obs:#1}}
\newcommand{\asmref}[1]{Assumption~\ref{asm:#1}}
\newcommand{\thref}[1]{Thread~\ref{th:#1}}
\newcommand{\trnref}[1]{Transaction~\ref{trn:#1}}
\newcommand{\grfref}[1]{Graph~\ref{graph:#1}}

\newcommand{\lineref}[1]{Line~\ref{lin:#1}}
\newcommand{\algoref}[1]{{Algorithm \ref{alg:#1}}}
\newcommand{\subsecref}[1]{SubSection~\ref{subsec:#1}}
\newcommand{\subsubsecref}[1]{SubSubSection~\ref{subsubsec:#1}}

\newcommand{\apnref}[1]{Appendix~\ref{apn:#1}}
\newcommand{\invref}[1]{Invariant~\ref{inv:#1}}

\newcommand{\specref}[1] {Specification~\ref{spec:#1}}

\newcommand{\Chapref}[1]{Chapter~\ref{chap:#1}}
\newcommand{\Secref}[1]{Section~\ref{sec:#1}}
\newcommand{\Figref}[1]{Figure~\ref{fig:#1}}
\newcommand{\Tabref}[1]{Table~\ref{tab:#1}}
\newcommand{\Stref}[1]{Step~\ref{step:#1}}
\newcommand{\Thmref}[1]{Theorem~\ref{thm:#1}}
\newcommand{\Lemref}[1]{Lemma~\ref{lem:#1}}
\newcommand{\Corref}[1]{Corollary~\ref{cor:#1}}
\newcommand{\Axmref}[1]{Proposition~\ref{axm:#1}}
\newcommand{\Defref}[1]{Definition~\ref{def:#1}}
\newcommand{\Eqref}[1]{(\ref{eq:#1})}
\newcommand{\Eqvref}[1]{Equivalence~(\ref{eqv:#1})}
\newcommand{\Ineqref}[1]{Inequality~(\ref{ineq:#1})}
\newcommand{\Exref}[1]{Example~\ref{ex:#1}}
\newcommand{\Propref}[1]{Property~\ref{prop:#1}}
\newcommand{\Obsref}[1]{Observation~\ref{obs:#1}}
\newcommand{\Asmref}[1]{Assumption~\ref{asm:#1}}
\newcommand{\Specref}[1]{Specification~\ref{spec:#1}}

\newcommand{\Lineref}[1]{Line~\ref{lin:#1}}
\newcommand{\Algoref}[1]{{\sf Algo$_{\ref{algo:#1}}$}}

\newcommand{\Apnref}[1]{Section~\ref{apn:#1}}
\newcommand{\Invref}[1]{Invariant~\ref{inv:#1}}
\newcommand{\Grfref}[1]{Graph~\ref{trn:#1}}


% environment for writing a proof

\newcommand{\proofsketch}[1][]{\noindent{\bf\boldmath
                         P\hspace{-0.25ex}roof~Sketch{#1}\/:\unboldmath}\hspace*{0.5em}}


\newcommand{\theqed}{$\Box$}

%\newcommand{\qed}{\hspace*{\fill}\theqed\\\vspace*{-0.5em}}

% In case the proof is immediately followed by the start of a new section

\newcommand{\nsqed}{\hspace*{\fill} \theqed}


% End of example

\newcommand{\eoe}{\qed}

% Renews the footnote command
%\renewcommand{\thefootnote}{\alph{footnote}}

\def\Nomega{\ensuremath{\neg\Omega}}
\def\Vomega{\ensuremath{\overrightarrow{\Omega}}}

\newcommand{\myparagraph}[1]{\vspace{1mm}\noindent\textbf{#1}}

%------------------------------------------------------------------------------------------------------------------
% Definitions for Weak Consistency paper
%------------------------------------------------------------------------------------------------------------------

\newcommand{\op} {operation\xspace}
\newcommand{\termop} {terminal operation\xspace}

\newcommand{\sble} {serializable\xspace}
\newcommand{\sbty} {serializability\xspace}
\newcommand{\lble} {linearizable\xspace}
\newcommand{\lbty} {linearizability\xspace}

\newcommand{\wf} {wait-free\xspace}

\newcommand{\CAS} {\textit{CAS}\xspace}
\newcommand{\mCAS} {\textit{mCAS}\xspace}
\newcommand{\cas} {CAS\xspace}

\newcommand{\cc} {correctness-criterion\xspace}

\newcommand{\rs}{rset\xspace}
\newcommand{\ws}{wset\xspace}


%\newcommand{\Fwrite}{\SetKwFunction{Fwrite}{ABCAS-append}}
%\newcommand{\Fview}{\SetKwFunction{Fview}{ABCAS-read}}

\newcommand{\Fwrite}{{\tt StickyCAS-append}\xspace}
\newcommand{\Fview}{{\tt StickyCAS-read}\xspace}
\newcommand{\BVCAS}{StickyCAS\xspace}
\newcommand{\bvcas}{StickyCAS\xspace}
\newcommand{\abcas}{StickyCAS\xspace}
%\Fwrite{}
%\Fview{}

\newcommand{\Skey} {Worm}
\newcommand{\skey} {worm}
\newcommand{\ie}{\textit{i.e., }}


\newcommand{\mth} {method\xspace}

\newcommand{\cons} {SM-ByzCons}
\newcommand{\sch} {scheduler\xspace}

\newcommand{\sbc} {Strong Byzantine Consensus\xspace}
\newcommand{\cvbc} {Common-value Byzantine Consensus\xspace}
\newcommand{\wbc} {Weak Byzantine Consensus\xspace}