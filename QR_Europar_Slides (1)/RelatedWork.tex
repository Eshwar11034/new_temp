% Slide: Related Work
\begin{frame}{Related Work \& Our Contribution}

	\begin{columns}[T,onlytextwidth]	
		\pause 
		\column{0.45\textwidth}
		\begin{block}{Advances in Optimization Solvers}
		  \begin{itemize}
		    \item Solvers like \textbf{IPOPT} and \textbf{SNOPT} are continuously evolving.
		    \item NLOPT's \textbf{SLSQP}, while popular, has lacked recent performance updates.
		    \item \textbf{PySLSQP} improved usability, but not core computational speed.
		  \end{itemize}
		\end{block}
		
		% ----- RIGHT: Existing Work in Parallel QR -----
		\column{0.5\textwidth}
		\begin{block}{Advances in Parallel QR Factorization}
		  \begin{itemize}
		    \item Significant research exists on parallelizing QR, often using task-based DAG scheduling.
		    \item Key works (Buttari et al., Baskaran et al.) have explored tiled methods and dynamic scheduling with priority queues.
		  \end{itemize}
		\end{block}	
	\end{columns}

	\pause 
\begin{alertblock}{The Gap \& Our Contribution}
    \begin{itemize}
      \item \textbf{The Challenge:} Achieving low-overhead, high-performance QR factorization for the \textbf{small-to-medium matrix sizes} common in iterative solvers.
      \item \textbf{Our Contribution:} Our lightweight scheduler is specifically designed for this scenario, delivering a \textbf{2-3x speedup} over mature libraries in this critical range.
    \end{itemize}
  \end{alertblock}

\end{frame}