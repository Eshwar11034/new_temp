% ------------------------------------------------------------------
% Slide: Barrier-Based Task Scheduling (concise version)
% ------------------------------------------------------------------
\begin{frame}{List Scheduling with Barriers}
	\pause 
	\begin{columns}[c,onlytextwidth]
	
	% ----- LEFT: Barrier logic ------------------------------------
	\column{0.50\textwidth}
	\begin{block}{The Barrier-Based Approach}
	  \begin{itemize}\setlength{\itemsep}{3pt}
	    \item Model QR as a graph of tasks ($T_{i,j}$).
	    \item \textbf{Barrier 1}: All threads wait after pivot task ($T_{i,i}$) is completed.
	    \item \textbf{Barrier 2}: All threads wait after all update tasks ($T_{i,j}$) finish before the next iteration.
	  \end{itemize}
	\end{block}
	
	% ----- RIGHT: Critical flaw text + figure ---------------------
	\pause	
	\column{0.45\textwidth}
	\begin{alertblock}{Why Barriers Hurt Performance}
		\begin{itemize}\setlength{\itemsep}{3pt}
		 \item Barriers force strictly synchronous execution.
		 \item Faster threads idle while the slowest thread finishes.
		 \item The idle time reduces overall throughput.
		\end{itemize}
	\end{alertblock}
	\end{columns}
	
	\pause
	\begin{columns}
	\column{0.70\textwidth}
	\begin{tikzpicture}[font=\tiny, every node/.style={transform shape}, scale=0.90]
		% Styles for different states
		\tikzset{pivot/.style={fill=pivotblue!70, draw=pivotblue!50!black, rounded corners=1pt}}
		\tikzset{work/.style={fill=updategreen!70, draw=updategreen!50!black, rounded corners=1pt}}
		\tikzset{wait/.style={fill=gray!40, draw=gray!60, rounded corners=1pt}}
		\tikzset{bar/.style={draw=depred, thick, dashed}}
		
		% Time Axis
		\draw[->] (0,0) -- (9.5,0) node[anchor=north west, font=\scriptsize] {Time};
		
		% Thread Labels
		\node[anchor=east, font=\scriptsize] at (-0.2, 2.5) {Thread 1};
		\node[anchor=east, font=\scriptsize] at (-0.2, 1.9) {Thread 2};
		\node[anchor=east, font=\scriptsize] at (-0.2, 1.3) {Thread 3};
		\node[anchor=east, font=\scriptsize] at (-0.2, 0.7) {Thread 4};
		
		% --- Iteration i with new, narrower coordinates ---
		\draw[pivot] (0.2, 2.3) rectangle node{$T_{i,i}$} (1.2, 2.7);
		\draw[wait]  (0.2, 1.7) rectangle (1.2, 2.1);
		\draw[wait]  (0.2, 1.1) rectangle (1.2, 1.5);
		\draw[wait]  (0.2, 0.5) rectangle (1.2, 0.9);
		\node[above=2pt, depred, font=\scriptsize] at (1.3, 2.8) {Barrier 1};
		\draw[bar] (1.3, 0.3) -- (1.3, 2.8);
		
		\draw[work] (1.4, 2.3) rectangle node{$T_{i,j}$} (4.5, 2.7);
		\draw[work] (1.4, 1.7) rectangle node{$T_{i,k}$} (6.5, 2.1);
		\draw[work] (1.4, 1.1) rectangle node{$T_{i,l}$} (8.0, 1.5); % SLOWEST
		\draw[work] (1.4, 0.5) rectangle node{$T_{i,m}$} (3.5, 0.9);
		
		\draw[wait] (4.5, 2.3) rectangle node{\textcolor{black}{IDLE}} (8.0, 2.7);
		\draw[wait] (6.5, 1.7) rectangle node{\textcolor{black}{IDLE}} (8.0, 2.1);
		\draw[wait] (3.5, 0.5) rectangle node{\textcolor{black}{IDLE}} (8.0, 0.9);
		
		\node[above=2pt, depred, font=\scriptsize] at (8.1, 2.8) {Barrier 2};
		\draw[bar] (8.1, 0.3) -- (8.1, 2.8);
		
		\draw[pivot] (8.2, 2.3) rectangle node{$T_{i+1,i+1}$} (9.2, 2.7);
	\end{tikzpicture}
	
	\column{0.25\textwidth}
	{\scriptsize \emph{Fig : Global barriers introduce idle time and squander parallel resources.}}
	\end{columns}

\end{frame}
