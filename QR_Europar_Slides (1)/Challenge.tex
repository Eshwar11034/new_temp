\begin{frame}{Why Focus on QR Factorization?}
	
	\pause
	\begin{columns}[T,onlytextwidth]
		\column{0.515\textwidth}
		\begin{block}{Runtime Cost}
		  \begin{itemize}\setlength{\itemsep}{6pt}
		    \item Each \textbf{SLSQP iteration} requires solving a linearized subproblem.  
		    \item This involves a \textbf{QR factorization} of the constraint Jacobian.  
		    \item QR calls appear repeatedly, making them the \textbf{dominant runtime cost}.  
		    \item Naïve parallel implementations rely on \textbf{global barriers}, which stall threads and waste resources.
		  \end{itemize}
		  \vfill
		\end{block}
	
		\pause
	
		\column{0.45\textwidth}
		\begin{block}{Opportunity}
		  The QR factorization step is not only the \textbf{bottleneck} in SLSQP,  
		  but also a prime opportunity for optimization:
		  \begin{itemize}\setlength{\itemsep}{6pt}
		    \item It has a \textbf{structured task graph}.  
		    \item Dependencies are clear and exploitable.  
		    \item Unlocking fine-grained parallelism can \textbf{reduce barrier overheads}.
		  \end{itemize}
		\end{block}
		\end{columns}
\end{frame}
