% Slide: Scaling with Threads (Throughput - Paper Version)
\begin{frame}{Results: Throughput Evaluation}

  \begin{columns}[c,onlytextwidth] % Vertically centered

    % ----- LEFT: The throughput graph, using the exact code from your paper -----
    \pause
    \column{0.6\textwidth}
    \begin{figure}
      \centering
      % NOTE: This is the PGFPlots code from your expts.tex file
      \begin{tikzpicture}[scale=0.9, transform shape]
        \begin{axis}[
            xlabel={Threads},
            ylabel={Execution Time (s)},
            % Using your paper's defined style
            myAxisStyle/.style={
                tick label style={font=\scriptsize},
                label style={font=\small},
                title style={at={(0.5,1.1)}, anchor=south, font=\small},
                grid=both,
                width=9cm, % Adjusted for slide width
                height=7cm
            },
            myAxisStyle,
            ymode=log,
            log basis y=10,
            xtick={4, 24, 44, 64, 84, 100},
            ytick={10, 100, 1000, 10000, 100000},
            legend style={at={(axis description cs:0.5,0.5)}, anchor=west, nodes={scale=0.7, transform shape}}
            ]
            % Barrier Data (Full dataset)
            \addplot[darkgreen, mark=o] coordinates {
                (4,60.32933) (8,37.08767) (12,29.13767) (16,24.59833)
                (20,22.211) (24,20.787) (28,20.160) (32,21.67867)
                (36,23.511) (40,21.980) (44,21.87833) (48,24.22767)
                (52,27.220) (56,27.414) (60,27.967) (64,28.367)
                (68,28.91533) (72,29.754) (76,30.85733) (80,32.056)
                (84,32.97067) (88,34.05467) (92,35.255) (96,36.73167)
                (100,38.23033)
            };
            \addlegendentry{Barrier};
            
            % Without Priority Data (Full dataset)
            \addplot[red, mark=square] coordinates {
                (4,19.068) (8,9.88267) (12,6.753) (16,5.495)
                (20,4.492) (24,3.996) (28,3.563) (32,3.33167)
                (36,3.169) (40,3.258) (44,3.030) (48,2.87633)
                (52,2.87967) (56,2.85433) (60,2.82233) (64,2.883)
                (68,2.971) (72,3.06733) (76,3.15033) (80,3.26467)
                (84,3.438) (88,3.65933) (92,3.891) (96,4.26467)
                (100,4.668)
            };
            \addlegendentry{Without Priority};
            
            % With Priority Data (Full dataset)
            \addplot[blue, mark=triangle] coordinates {
                (4,19.20933) (8,10.104) (12,7.13767) (16,6.22233)
                (20,5.67667) (24,4.84067) (28,4.59033) (32,4.56433)
                (36,4.55167) (40,4.48067) (44,4.507) (48,4.11267)
                (52,3.804) (56,4.03433) (60,3.80433) (64,3.72833)
                (68,3.518) (72,3.45633) (76,3.65533) (80,3.98367)
                (84,4.212) (88,4.07933) (92,4.493) (96,4.891)
                (100,5.32067)
            };
            \addlegendentry{With Priority};
        \end{axis}
      \end{tikzpicture}
    \end{figure}

    % ----- RIGHT: Concise text block for analysis -----
    \column{0.4\textwidth}
    \begin{block}{The Experiment}
      We measure execution time on a fixed-size matrix ($8192 \times 8192$) while increasing the number of threads.
    \end{block}
    
    \pause
    \begin{alertblock}{Key Findings}
      \begin{itemize}
        \item Our two-queue schedulers show strong scaling, with performance improving significantly as threads are added.
        \item The barrier-based method scales poorly due to synchronization overhead, becoming slower after an initial improvement.
      \end{itemize}
    \end{alertblock}

  \end{columns}
\end{frame}