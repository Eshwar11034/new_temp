% ------------------------------------------------------------------
% Slide: What the Kernels Actually Do  (professional version)
% ------------------------------------------------------------------
\begin{frame}{What Do The Kernels Actually Do?}
	
	\pause 
 	\begin{columns}[T,onlytextwidth]
		% -------- Pivot Kernel (LEFT) -----------------------------------
		\column{0.48\textwidth}
		\begin{block}{1. Pivot Kernel – Compute the Transformation}
			\begin{itemize}\setlength{\itemsep}{3pt}
			  \item Examines the current pivot row.
			  \item Computes a transformation to introduce zeros above the diagonal.
			  \item Updates the row in-place and outputs transformation parameters.
			\end{itemize}
		\end{block}
		
		% mini-figure (pivot BEFORE → AFTER)
		\centering
		\begin{tikzpicture}[font=\scriptsize, every node/.style={transform shape}, scale=0.9]
			\node at (-1.9, 1) {$\begin{pmatrix} d & c & g \end{pmatrix}$};
			\node at (-1.9,  0.35) {$\begin{pmatrix} a & e & h \end{pmatrix}$};
			\node at (-1.9, -0.45) {$\begin{pmatrix} b & c & f \end{pmatrix}$};
			\node at ( 1.9, 1) {$\begin{pmatrix} d' & \mathbf{0} & \mathbf{0} \end{pmatrix}$};
			\node at ( 1.9,  0.35) {$\begin{pmatrix} a & e & h \end{pmatrix}$};
			\node at ( 1.9, -0.45) {$\begin{pmatrix} b & c & f \end{pmatrix}$};
			\draw[-{Stealth}, pivotblue, thick] (-0.6,0.85) -- (0.6,0.85)
		      node[midway, above] {\tiny \texttt{update\_pivot\_row}};
		\end{tikzpicture}
		
		\pause
		% -------- Update Kernel (RIGHT) ---------------------------------
		\column{0.48\textwidth}
		\begin{block}{2. Update Kernel – Apply the Transformation}
			\begin{itemize}\setlength{\itemsep}{3pt}
			  \item Receives parameters from the pivot kernel.
			  \item Applies the transformation to each trailing row.
			  \item Enables parallel execution across multiple rows.
			\end{itemize}
		\end{block}
		
		% mini-figure (two rows BEFORE → AFTER)
		\centering
		\begin{tikzpicture}[font=\scriptsize, every node/.style={transform shape}, scale=0.9]
			% before rows
			\node at (-2.1, 1) {$\begin{pmatrix} d' & \mathbf{0} & \mathbf{0} \end{pmatrix}$};
			\node at (-2.1,  0.35) {$\begin{pmatrix} a & e & h \end{pmatrix}$};
			\node at (-2.1, -0.45) {$\begin{pmatrix} b & c & f \end{pmatrix}$};
			% after rows
			\node at (3.2, 1) {$\begin{pmatrix} d' & \mathbf{0} & \mathbf{0} \end{pmatrix}$};
			\node at ( 3.2,  0.35) {$\begin{pmatrix} a' & e' & h' \end{pmatrix}$};
			\node at ( 3.2, -0.45) {$\begin{pmatrix} b' & c' & f' \end{pmatrix}$};
			% arrows
			\draw[-{Stealth}, updategreen, thick] (-0.8,  0.35) -- (1.5,  0.35)
			node[midway, below] {\tiny \texttt{update\_trailing\_non-pivot\_row}};
			\draw[-{Stealth}, updategreen, thick] (-0.8, -0.45) -- (1.5, -0.45);
		\end{tikzpicture}
	\end{columns}

	\pause
	\vspace{2mm}
	{\scriptsize \emph{The pivot kernel computes transformation parameters, which the update kernel applies to construct the upper-triangular matrix required by SLSQP.}}

\end{frame}