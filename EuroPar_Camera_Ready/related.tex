% The NLOPT\cite{NLopt} library, particularly its implementation of the SLSQP\cite{SLSQP} algorithm, is widely recognized for its effectiveness in solving nonlinear optimization problems. Notably, SLSQP is an open-source algorithm that allows users to integrate and customize it according to their optimization needs. It is particularly valued for its numerical stability. While SLSQP remains a popular choice, it has not seen significant advancements compared to other contemporary solvers, such as IPOPT\cite{biegler2009large} and SNOPT\cite{gill2005snopt}, which have continuously evolved to offer enhanced features and capabilities.

% SNOPT, a commercial optimization software, and IPOPT, which is also open-source like SLSQP, have both demonstrated ongoing improvements and the implementation of new techniques that keep them competitive in the optimization landscape\cite{joshy2024pyslsqp}. The need for enhanced performance in existing SLSQP implementations prompted the development of the PySLSQP\cite{joshy2024pyslsqp} package, introduced by Joshy et al., which enhances the utility of the SLSQP algorithm by addressing limitations in current formulations, particularly those found in NLOPT. PySLSQP effectively bridges Python with the original Fortran code, enabling users to customize and modify the algorithm with greater ease.

% In the realm of parallel computing, efficient QR decomposition techniques have been explored extensively in the literature. Notably, Buttari et al.\cite{buttari2008parallel} introduced a Parallel Tiled QR factorization method that constructs a DAG for scheduling purposes. Baskaran et al. work further contributes to this field by discussing compiler-assisted dynamic scheduling using lock-based priority queues\cite{baskaran2009compiler}. Building on Baskaran's innovations, Roshan et al. developed dynamic scheduling techniques that implement lock-free priority queues for scheduling QR factorization algorithm\cite{dathathri2016compiling}.

% Our work extends existing approaches by enhancing the NLOPT SLSQP solver by integrating advanced asynchronous parallel algorithms. Unlike prior implementations that rely primarily on traditional synchronous strategies, our method incorporates novel parallel scheduling techniques, drawing inspiration from dynamic and lock-free approaches in QR factorization, to improve both performance and scalability.

% There have been many prior works on efficiently doing DAG scheduling~\cite{DAG-scheduling,dynamic-scheduling}. In this paper, we use a variation of DAG scheduling to efficiently perform the interdependent tasks of QR decomposition.

The NLOPT\cite{NLopt} library's SLSQP\cite{SLSQP} algorithm is a recognized open-source, numerically stable solver for nonlinear optimization, valued for its customizability. However, unlike continuously evolving contemporaries such as IPOPT\cite{biegler2009large} (open-source) and SNOPT\cite{gill2005snopt} (commercial), SLSQP has seen limited recent advancements. To address this, Joshy et al. introduced the PySLSQP\cite{joshy2024pyslsqp} package, enhancing SLSQP's utility by bridging Python with the original Fortran code, thereby facilitating easier modification and addressing limitations in current formulations, particularly those in NLOPT.
Concurrently, significant progress has been made in parallel QR decomposition. Buttari et al.\cite{buttari2008parallel} introduced a Parallel Tiled QR factorization method using a DAG for scheduling. Baskaran et al.\cite{baskaran2009compiler} contributed with compiler-assisted dynamic scheduling using lock-based priority queues, and building on this, Roshan et al. (Dathathri et al.)\cite{dathathri2016compiling} developed dynamic scheduling techniques with lock-free priority queues for QR factorization.
This research extends existing approaches by enhancing the NLOPT SLSQP solver through the integration of advanced asynchronous parallel algorithms. Departing from traditional synchronous strategies, this method incorporates novel parallel scheduling techniques inspired by the dynamic and lock-free approaches developed for QR factorization, aiming to improve both performance and scalability. For managing the interdependent tasks within QR decomposition, this work employs a variation of DAG scheduling, drawing upon established efficient DAG scheduling methodologies~\cite{DAG-scheduling}.