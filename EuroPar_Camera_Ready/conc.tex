This study integrates a highly parallel QR factorization into NLOPT’s SLSQP routine by decomposing the factorization into numerous independent micro-tasks and coordinating them with an asynchronous, dependency-aware DAG scheduler. The resulting workflow markedly cuts computational overhead and achieves consistent speed-ups across a broad suite of benchmark problems, underscoring the value of fine-grained parallelism in nonlinear constrained optimization.
In future work, we will substitute the classical Householder QR with a tiled implementation \cite{buttari2008parallel}\cite{baskaran2009compiler}\cite{dathathri2016compiling}. Tiling will expose even finer parallel granularity, facilitate NUMA-aware task placement, and enable closed-form selection of tuning parameters $\alpha$ and $\beta$, thereby eliminating costly parameter sweeps. Moreover, tiling naturally reduces queue-transfer contention between wait and execution queues by improving spatial locality and minimizing remote memory traffic.
We further intend to generalize this tiling and scheduling strategy to the full suite of linear-algebra kernels invoked by SLSQP—such as rank-update, triangular solve, and Cholesky-related operations—ultimately delivering a solver that scales robustly on diverse multicore and many-core architectures.

%In this study, we have successfully developed and integrated parallel techniques for QR factorization within the SLSQP algorithm of the NLOPT library. Our approach has significantly reduced computational overhead by decomposing the QR factorization into smaller independent tasks and employing an asynchronous DAG scheduling mechanism. The empirical results underscore the transformative potential of parallel computing techniques in improving optimization efficiency in diverse engineering and scientific applications.

%Looking ahead, we're excited about several promising directions to boost the performance of the NLOPT SLSQP algorithm even further. For starters, we plan to replace the traditional Householder QR method with a Tiled QR algorithm, inspired by \cite{buttari2008parallel}\cite{baskaran2009compiler}\cite{dathathri2016compiling}. We believe this change will allow us to tap into advanced scheduling techniques more effectively, boosting computational efficiency. We also believe that only after implementing a proper tiled version of the proposed algorithm, we can provide mathematical formulations to obtain the best values of the parameters $\alpha$ and $\beta$ for a given input size without having to perform a parameter sweep in advance. In this work, our primary focus was on optimizing task graph scheduling, while aspects such as contention during task transfers between the wait queue and the main queue—particularly in a NUMA setting—were not explicitly addressed. We believe that these issues, along with improved task locality, can be more effectively tackled once a tiled version of the algorithm is in place. Finally, we aim to optimize other linear algebra kernels used in SLSQP, which should help extend these performance gains beyond just the QR factorization. 

%Ultimately, we aim to integrate efficient parallel techniques into every key linear algebra operation in SLSQP. We believe that by doing so, we can boost performance even further while making the solver more adaptable to a wide range of optimization challenges. We hope these efforts will help drive forward high-performance optimization tools that can keep up with the growing computational needs of modern science and engineering.
